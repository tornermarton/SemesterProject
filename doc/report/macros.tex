%%%%%%%%%%%%%%%%%%%%%%%%%%%%%%%%%%%%%%%%%%%%%%%%%%%%%%%%%%%%%%%%%%%
% Ezt ne piszkáld!!!!
%%%%%%%%%%%%%%%%%%%%%%%%%%%%%%%%%%%%%%%%%%%%%%%%%%%%%%%%%%%%%%%%%%%
\pagestyle{myheadings} % legyen fejléc 

\newcommand{\semesterprojecttitle}{
  \begin{center}
    \huge{\textbf{Semester project report}}
    
    \smallskip
    \small{Department of Telecommunications and Media Informatics}
  \end{center}
  \bigskip
} 

% Argumentumok: #1=Név, #2=Neptunkód, #3=szakirány, #4=email, #5 konzulens-1, #6 konzulens-1-email, #7 konzulens-2, #8 konzulens-2-email
\newcommand{\semesterprojectauthor}[6]{

\begin{center}
  \begin{tabular}{ l c }
  Author:         & \textbf{#1}  \\
  Neptun:         & \textbf{\texttt{#2}}  \\
  Specialization: & \textbf{#3}  \\
  E-mail address: & \href{mailto:#4}{\textbf{#4}}  \\
  Supervisor:     & \textbf{#5}  \\
  E-mail addres:  & \href{mailto:#6}{\textbf{#6}} \\
  
  \end{tabular}
\end{center}
\bigskip

}

% % Argumentumok: #1=Név, #2=email
% \newcommand{\konzulens}[2]{
%   \noindent\textbf{Konzulens:} #1 
%   \newline\emph{Email cím:}\/ \href{mailto:#2}{#2}
%   \newline
% 
% }

% Argumentumok: #1=Tanév (xxxx/xx alakban, #2=félév (pont nélkül)
\newcommand{\semester}[2]{
  \large\textbf{#1 #2. semester}
}

% Argumentumok: #1=téma címe 
\newcommand{\tasktitle}[1]{
  \begin{center}
    \LARGE\textbf{Title: #1}
    \bigskip
  \end{center}
}

% Argumentumok: #1=téma részletei 
\newcommand{\taskdescription}[1]{
\begin{center}
  \Large\textbf{Task} 
\end{center}
#1
\newline
\bigskip
}

% A fejezetek közé beágyazott irod.jegyzék
\def\references#1{\renewcommand{%
\baselinestretch}{1}\list
 {\small [\arabic{enumi}]}{\settowidth\labelwidth{[#1]}\leftmargin\labelwidth
 \advance\leftmargin\labelsep
 \usecounter{enumi}}
 \def\newblock{\small \hskip .11em plus .33em minus .07em}
 \sloppy\clubpenalty4000\widowpenalty4000
 \sfcode`\.=1000\relax}
\let\endreferences=\endlist%


%%% Local Variables: 
%%% mode: latex
%%% TeX-master: "template"
%%% End: 
